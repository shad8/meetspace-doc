\section{Wykorzystane technologie}
  \subsection{Ruby}
  \subsection{Ruby on Rails}
  \subsection{Git}
  \subsection{RVM}
  \subsection{Narzędzia do testowania}
    \subsubsection{Cucumber}
    \subsubsection{RSpec}
  \subsection{Wykorzystane gemy}
    Gem - jest to paczka napisana w języku Ruby, której zadaniem jest rozszerzenie funkcjonalności aplikacji. Do wyszukiwania najnowszych wersji wykorzystuje się RubyGems.org\footnote{Wyszukiwarka gemów \url{http://rubygems.org}}. W przypadku Ruby on Rails istnieje specjalny plik konfiguracyjny, \emph{Gemfile.rb}, który przechowuje liste wykorzystywanych gemów.

    W naszej aplikacji użyliśmy między innymi:
    \begin{itemize}
      \item \emph{Devise} \\ Zapewnia autoryzacje i autentykacje użytkownika w obrębie aplikacji. Obsługuje logowanie oraz rejestrację wraz z wysyłaniem potwierdzeń na adres mailowy i reset hasła.
      \item \emph{Geocoder} \\ Na podstawie podanego adresu określa współrzędne geograficzne.
      \item \emph{Omniauth-facebook} \\ Wspiera komunikację pomiędzy naszą aplikacją a API Facebook'a.
      \item \emph{i18n} \\ Umożliwia dodawanie tłumaczeń, dzięki czemu aplikacja w łatwy sposób może stać się wielojęzyczną.
    \end{itemize}
  \subsection{Biblioteki JavaScript}
    \begin{itemize}
      \item \emph{jQuery} \\ Biblioteka JavaScript ułatwiająca manipulaję drzewem DOM\footnote{Document Object Model - reprezantacja złożonych dokumentów HTML. \cite{html5_css3}}. Umożliwia tworzenie animacji, wspomaga zarządzanie asynchronicznymi zapytaniami do serwera. Dzięki jej implementacji możliwe jest korzystanie z wielu stworzonych już pluginów.
      \item \emph{Bootstrap Datepicker} \\ Oparta o Bootstrap biblioteka, wyświetlająca na stronie ładny i intuicyjny kalendarz.
      \item \emph{Load Image} \\ Prosta biblioteka umożliwiająca użytkownikowi ładowanie obrazka bezpośrednio na stronie.
      \item Google Maps API
    \end{itemize}
  \subsection{Pozostałe technologie}
    \begin{itemize}
      \item \emph{Bootstrap}
        Framwork CSS, rozwijany przez firmę Twitter. Udostępnia gotowe klasy CSS i rozwiązania JavaScript, dzięki czemu pisanie responsywnych layoutów naitemdzenia mobilne i nie tylko, jest znacznie szybsze i mniej bolesne.
      \item \emph{CoffeScript}
      \item \emph{SASS}
    \end{itemize}
  \subsection{Narzędzia pomocnicze}
    \begin{itemize}
      \item \emph{Sublime Text 3}
      \item \emph{Narzędzia developerskie Google Chrome}
    \end{itemize}
