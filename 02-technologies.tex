\section{Wykorzystane technologie}
  \subsection{Ruby}
  Ruby jest w pełni obiektowym, wysokopoziomowym językiem programowania. Jego twórca, Yukihiro “Matz” Matsumoto, chciał stworzyć język jeszcze bardziej obiektowy niż Python, dlatego każdy fragment informacji może uzyskać swoje właściwości i metody. Po za obiektowością, Rubiego cechuje też prosta składnia, ułożona w sposób umożliwiający pisanie samokomentującego się i czytelnego kodu. Programiści piszący w tym języku, kierują się dwoma ważnymi zasadami: DRY\footnote{Don't repeat yourself} oraz KISS\footnote{Keep it simple, stupid}, które mają za zadanie zmusić do wykorzystywania już wcześniej napisanego kodu jak i pisania go w sposób najmożliwiej prosty. Ponadto, jest to język niezwykle elastyczny, ponieważ pozwala użytkownikom dowolnie modyfikować jego składowe. Programista może nadpisać jakiś moduł i dostosować go do swoich potrzeb. Ruby powstał w 1995 roku, lecz dopiero w 2005 roku zyskał na popularności za sprawą frameworka Ruby on Rails, który został w nim napisany. Dzisiaj osoby piszące w Rubym to jedni z najlepiej zarabiających programistów w USA\footnote{\url{http://www.pb.pl/3945225,84037,te-jezyki-programowania-daja-najlepiej-zarobic-w-usa}}.

  \subsection{Ruby on Rails}
  Jest to framework, platforma programistyczna, pozwalająca na szybkie tworzenie stron i aplikacji internetowych. Została napisana przez duńskiego programistę Davida Heinemeiera Hanssona, zwanego potocznie DHH. Oparta jest o wzorzec architektoniczny MVC\footnote{Model View Controller, więcej informacji w rozdziale poświęconym wykorzystanym wzorcom projektowym w pracy.}. \\
  Tim O'Reilly, założyciel O'Reilly Media, powiedział:
  \begin{quote}
    \emph{“Ruby on Rails jest przełomem w dziedzinie programowania aplikacji internetowych.
    Potężne aplikacje, których tworzenie do tej pory zabierało tygodnie czy miesiące, są teraz tworzone dosłownie w kilka dni.”}
  \end{quote}
  W Ruby on Rails panuje zasada \emph{Convention over configuration}, konwencja ponad konfiguracją, co znaczy, że nie trzeba się przejmować skomplikowanymi plikami konfiguracyjnymi, tylko wystarczy postępować według przyjętych przez twórców schematów. Railsy mają wbudowany serwer lokalny, co pozwala na szybkie testowanie aplikacji, bez zbędnego i czasochłonnego umieszczania kodu na zewnętrznej maszynie. Dużym plusem jest również możliwość uruchomienia aplikacji w różnym środowisku, do wyboru mamy:
  \begin{itemize}
    \item developerskie - domyślne środowisko, w którym programista pisze aplikację,
    \item testowe - wykorzystywane do testowania aplikacji,
    \item produkcyjne - aplikacja zachowuje się wtedy jakby była już umieszczona na serwerze
  \end{itemize}
  To wszystko sprawia, że klientowi można oddać gotowy, przetestowany produkt, bez żadnych niespodzianek.


  \subsection{Git}
  To rozproszony system kontroli wersji, bardzo pomocny w projektach, przy których pracuje kilka osób. Dzięki niemu widać wszystkie zmiany wcześniej wprowadzane i w łatwy sposób można wrócić do wcześniejszych wersji. Użytkownik posługuję się tzw. commitami. Jest to nic innego jak zatwierdzenie zmian w plikach. Podobnym narzędziem do Gita jest SVN. Różnica m.in. polega na tym, że Git kopiuje całe repozytorium na komputer, programista zatwierdza zmiany lokalnie i dopiero na koniec wrzuca je na serwer. W przypadku SVN całość trzymana jest na serwerze. Różnice między tymi dwoma technologiami pokazuje poniższy obrazek.

  \textit{tu będzie obrazek}
  % \begin{figure}
  %   \centering
  %   \includegraphics[scale=0.47]{images/gitsvn.png}
  %   \caption{Różnica w działaniu pomiędzy Git i SVN}
  % \end{figure}

  \subsection{RVM}
  Czyli Ruby Version Manager. Jest to podstawowe narzędzie do pracy z językiem Ruby. Ruby jest ciągle rozwijany i każdego miesiąca wychodzą nowe poprawki, usprawnienia czy funkcjonalności. RVM dla każdego projektu tworzy osobne, niezależne, odizolowane środowisko programistyczne, tzw. gemset. Jest to nic innego jak zbiór gemów wykorzystywanych w danym projekcie. Dzięki temu można stworzyć różne aplikacje oparte o różne wersje Rubiego, które korzystają z różnych wersji gemów i nie kolidują ze sobą.

  \subsection{Narzędzia do testowania}
    \begin{itemize}
      \item \emph{Cucumber}
      \item \emph{RSpec}
    \end{itemize}

  \subsection{Wykorzystane gemy}
    Gem - jest to paczka napisana w języku Ruby, której zadaniem jest rozszerzenie funkcjonalności aplikacji. Do wyszukiwania najnowszych wersji wykorzystuje się RubyGems.org\footnote{Wyszukiwarka gemów \url{http://rubygems.org}}. W przypadku Ruby on Rails istnieje specjalny plik konfiguracyjny, \emph{Gemfile.rb}, który przechowuje liste wykorzystywanych gemów.

    W naszej aplikacji użyliśmy między innymi:
    \begin{itemize}
      \item \emph{Devise} \\ Zapewnia autoryzacje i autentykacje użytkownika w obrębie aplikacji. Obsługuje logowanie oraz rejestrację wraz z wysyłaniem potwierdzeń na adres mailowy i reset hasła.
      \item \emph{Geocoder} \\ Na podstawie podanego adresu określa współrzędne geograficzne.
      \item \emph{Omniauth-facebook} \\ Wspiera komunikację pomiędzy naszą aplikacją a API Facebook'a.
      \item \emph{i18n} \\ Umożliwia dodawanie tłumaczeń, dzięki czemu aplikacja w łatwy sposób może stać się wielojęzyczną.
    \end{itemize}
  \subsection{Biblioteki JavaScript}
    \begin{itemize}
      \item \emph{jQuery} \\ Biblioteka JavaScript ułatwiająca manipulaję drzewem DOM\footnote{Document Object Model - reprezantacja złożonych dokumentów HTML. \cite{html5_css3}}. Umożliwia tworzenie animacji, wspomaga zarządzanie asynchronicznymi zapytaniami do serwera. Dzięki jej implementacji możliwe jest korzystanie z wielu stworzonych już pluginów.
      \item \emph{Bootstrap Datepicker} \\ Oparta o Bootstrap biblioteka, wyświetlająca na stronie ładny i intuicyjny kalendarz.
      \item \emph{Load Image} \\ Prosta biblioteka umożliwiająca użytkownikowi ładowanie obrazka bezpośrednio na stronie.
      \item \emph{Google Maps API}
    \end{itemize}
  \subsection{Pozostałe technologie}
    \begin{itemize}
      \item \emph{Bootstrap}
        Framwork CSS, rozwijany przez firmę Twitter. Udostępnia gotowe klasy CSS i rozwiązania JavaScript, dzięki czemu pisanie responsywnych layoutów naitemdzenia mobilne i nie tylko, jest znacznie szybsze i mniej bolesne.
      \item \emph{CoffeScript}
      \item \emph{SASS}
    \end{itemize}
  \subsection{Narzędzia pomocnicze}
    \begin{itemize}
      \item \emph{Sublime Text 3}
      \item \emph{Narzędzia developerskie Google Chrome}
    \end{itemize}
