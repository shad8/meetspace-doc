\section{Testy}
	\label{testy}
    Technologie internetowe rozwijają się w zastraszającym tempie, co za tym idzie, wymagania klientów również. Wiele firm kładzie duży nacisk na testowanie swoich aplikacji.

  	Programiści często wychodzą z założenia, że ,,najpierw kod, później testy". Ma to oczywiście swoje wady i zalety. Zaletą niewątpliwie jest czas realizacji. Wynika to z tego, że w pierwszej kolejności pisze się daną funkcjonalność i nie zastanawia się nad tym jak napisać do niej test. Po skończeniu przychodzi czas na napisanie testów do niej. Wygląda to mniej więcej tak, że pisze się jeden lub dwa testy, które sprawdzą czy funkcjonalność w ogóle działa i dodatkowo kilka skrajnych przypadków w których może się wysypać. Doprowadza to do tego, że cały kod jest pokryty testami w bardzo małym stopniu, więc tak naprawdę w każdej chwili może zajść sytuacja, w której przestanie to działać tak jak powinno. Niestety w takim przypadku tracimy cenne godziny na szukanie błędu i jego eliminację.

  	My w swojej pracy przyjęliśmy zupełnie inne podejście, mianowicie ,,najpierw testy, później kod". Owszem, czas pisania znacznie się wydłuża, ale kod jest w dużo większym stopniu pokryty testami, dzięki czemu oszczędzamy sporo godzin przy diagnozie danego błędu. Kolejną zaletą takiego podejścia jest to, że napisany w ten sposób kod robi dokładnie to czego oczekujemy. Piszemy test i spełniamy go w najprostszy możliwy sposób. W ten sposób na jedną metodę przypada kilka lub nawet kilkanaście testów, ale kiedy któryś z nich się nie spełni, wiadomo od razu gdzie szukać przyczyny.

  \newpage

  \subsection{Techniki tworzenia oprogramowania}
    \subsubsection{TDD}
    Test Driven Development\cite{tdd} - technika tworzenia oprogramowania, sterowana przez testy. Polega na wielokrotnym powtarzaniu 3 kroków do momentu ukończenia funkcjonalności:
    \begin{enumerate}
      \item Napisanie możliwie najprostszego testu jednostkowego, który ma sprawdzać kod pisany w kroku 2.
      \item Implementacja kodu. Kod powinien być napisany w taki sposób, aby spełnić założenia testu, nic ponad to. Test powinien zakończyć się sukcesem i nie naruszać pozostałych testów
      \item Refaktoryzacja. Doprowadznie kodu do stanu, w którym spełnia przyjęte normy i standardy prostego oraz czytelnego kodu\cite{scs}.
    \end{enumerate}

    Postępowanie według tego schematu, zmusza programistę do wcześniejszego przemyślenia funkcjonalności, którą ma napisać.
    wady i zalety.


    \subsubsection{BDD}

	\newpage
  \subsection{Testy integracyjne}
  \subsection{Testy jednostkowe}
