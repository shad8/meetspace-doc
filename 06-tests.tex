\section{Testy}
  \label{testy}
    Technologie internetowe rozwijają się bardzo szybko, a co za tym idzie, wymagania klientów również. Wiele firm kładzie duży nacisk na testowanie swoich aplikacji.

    Programiści często wychodzą z założenia, że ,,najpierw kod, później testy". Ma to oczywiście swoje wady i zalety. Zaletą niewątpliwie jest czas realizacji. W pierwszej kolejności dostarcza się funkcjonalności i nie zastanawia się nad tym jak napisać do nich testy. Pod koniec procesu produkcji tworzy się kilka testów sprawdzających zachowanie aplikacji. Doprowadza to do tego, że cały kod aplikacji pokryty jest testami w bardzo niewielkim stopniu. W każdej chwili może zajść sytuacja, w której aplikacja przestanie działać zgodnie z założeniami. Niestety w takim przypadku tracimy cenne godziny na szukanie błędów i ich eliminację.

    Podczas tworzenia aplikacji Meetspace przyjęliśmy zupełnie inne podejście - „najpierw testy, później kod”. Czas pisania aplikacji znacznie się wydłużył. Osiągnęliśmy jednak bardzo dobre pokrycie kodu testami. \\ \\
    (dać tu wykres STATS - metryki i krótko omówić go, gdyż jest tam duży skok) \\ \\
    Dzięki napisanym testom oszczędziliśmy sporo godzin przy diagnozowaniu błędów aplikacji. Kolejną zaletą obranego podejścia jest to, że napisany w ten sposób przez nas kod robi dokładnie to czego oczekujemy. Każdy test spełniany jest w najprostszy możliwy sposób. Na jedną metodę przypada kilka lub nawet kilkanaście testów. Jeżeli któryś z nich się nie spełni, wiadomo od razu gdzie szukać przyczyny.

  \subsection{Techniki tworzenia oprogramowania}
    Techniki tworzenia oprogramowania są niezwykle ważne w całym procesie powstawania kodu. Pomagają efektywnie zarządzać zasobami zespołu, czasem, zadaniami i celami, które stoją przed \emph{team'em}.

    Poniżej krótko objaniśmy te techniki, z których korzystaliśmy.
    \index{TDD}
    \begin{itemize}
      \item Test Driven Development\cite{tdd} \\
        Technika tworzenia oprogramowania, sterowana przez testy. Polega na wielokrotnym powtarzaniu 3 kroków do momentu ukończenia funkcjonalności:
        \begin{enumerate}
          \item Napisanie możliwie najprostszego testu jednostkowego, który ma sprawdzać kod pisany w kroku 2.
          \item Implementacja kodu. Kod powinien być napisany w taki sposób, aby spełnić założenia testu, nic ponad to. Test powinien zakończyć się sukcesem i nie naruszać pozostałych testów
          \item Refaktoryzacja. Doprowadzenie kodu do stanu, w którym spełnia przyjęte normy i standardy prostego oraz czytelnego kodu\cite{scs}.
        \end{enumerate}

        Postępowanie według tego schematu, zmusza programistę do wcześniejszego przemyślenia funkcjonalności, którą ma napisać.

        Wady:
        \begin{itemize}
          \item Wydłuża się czas pisania aplikacji, zwłaszcza w początkowej fazie wdrażania tej techniki. Jednak wraz z ilością napisanych testów jednostkowych, rośnie wydajność ich pisania.
          \item Wraz z rozwojem aplikacji rośnie również ilość testów. Dopóki funkcjonalności są dopisywane, problem nie istnieje. Zmiana funkcjonalności narzuca zmiany po stronie istniejących do niej testów.
        \end{itemize}

        Zalety:
        \begin{itemize}
          \item Główną zaletą tej metodologii jest szybkość diagnozowania błędów. Dzięki temu oszczędzamy mnóstwo czasu. Bez testów musielibyśmy poświęcić go na dogłębną analizę kodu.
          \item Jeżeli testy są pisane w odpowiedni sposób, mogą stanowić bardzo dobrą dokumentację. Wystarczy zajrzeć do testu i widać czego programista oczekiwał od konkretnej metody.
          \item Bardziej przemyślany kod. W pierwszej kolejności musimy się zastanowić co tak naprawdę ma zawierać dana funkcjonalność, żeby móc od niej to wyegzekwować.
        \end{itemize}

        \begin{figure}
          \centering
          \includegraphics[scale=0.47]{images/test_cycle.png}
          \caption{Schemat postępowania w TDD}
        \end{figure}

      \newpage
      \index{BDD}
      \item Behaviour Driven Development \\
        Technika stworzona przez Dana Northa w 2003 roku. Sam autor powiedział:
        \begin{quote}
          \emph{„Behaviour-driven Development polega na tworzeniu oprogramowania przez opisywanie jego zachowania, z perspektywy jego udziałowców.”}
        \end{quote}

        Dzięki takiej metodyce testowania programiści wychodzą naprzeciw klientowi, starają się go zrozumieć i spełnić jego wymagania.

        Oczekiwania klienta są zapisywane w postaci krótkich historyjek, scenariuszy.
        Podczas ich tworzenia korzysta się z trzech głównych słów kluczowych:
          \begin{itemize}
            \item Podane (\emph{Given}) - opisuje stan początkowy aplikacji
            \item Kiedy (\emph{Wen}) - wszystkie kroki jakie klient/użytkownik musi wykonać aby osiągnąć funkcjonalność
            \item Wtedy (\emph{Then}) - stan końcowy aplikacji, czyli oczekiwany efekt po przejściu przez wcześniejsze kroki
          \end{itemize}

        \emph{Test driven development} to testowanie na poziomie kodu, \emph{behaviour driven development} to testowanie na poziomie aplikacji. Dlatego jeśli korzysta się z obu technik, zaczyna się od BDD, schodząc coraz ,,niżej'', aż do TDD. Wtedy mamy pewność, że konkretna funkcjonalność jest należycie przetestowana i spełnia oczekiwania oraz wymogi klienta.
    \end{itemize}

  \subsection{Testy integracyjne}
    \index{Cucumber}
    Najważniejszą funkcjonalnością w naszej aplikacji jest kreowanie nowego wydarzenia. Poniżej przedstawimy proces tworzenia testu integracyjnego.

    \input{source_codes/cucumber}
  \subsection{Testy jednostkowe}
    \index{Rspec}
    Testy jednostkowe weryfikują działanie pojedynczych elementów(jednostek) programu, poszczególnych metod czy relacji między obiektami. Taki test wykonuje dany fragment kodu i porównuje otrzymane wyniki z tymi oczekiwanymi. Każdy z testów jest wykonywany w osobnym, odizolowanym środowisku, dzięki czemu jest niezależny i nie wpływa na pozostałe.

    Aby pokazać ideę testów jednostkowych wykorzystamy model Event, główny trzon aplikacji.

Zaczynamy od wskazania klasy, która będzie testowana. Typ nie jest obowiązkowy, ale RSpec daje nam kilka predefiniowanych typów, m.in. \emph{controller}, \emph{model} czy \emph{routing} z czego warto korzystać. Oczywiście można testować też klasy, które nie należą do żadnego, ale korzystanie z nich znacznie przyspiesza wykonywanie testów.

\begin{code}
	\lstinputlisting[linerange={3-4}, firstnumber=1]{../meetspace/spec/models/event_spec.rb}
\end{code}\\

W drugiej linijce zadeklarowana jest zmienna \emph{event}, która przechowuje obiekt opisywanej klasy. RSpec wykorzystuje tzw. leniwe deklarowanie (\emph{lazy loading}). Zmienna, w tym przypadku obiekt, jest tworzona w momencie natrafienia na nią podczas wykonywania pliku.

\begin{code}
	\lstinputlisting[linerange={41-49}, firstnumber=3]{../meetspace/spec/models/event_spec.rb}
\end{code}\\

Słowo ,,\emph{Describe}'' służy do tworzenia osobnego bloku, co zwiększa czytelność i pozwala szybko zorientować się w pliku. Opisy bloków i testów są o tyle ważne, że tworzą swoistego rodzaju dokumentację. Na pierwszy rzut oka widać jakie testy znajdują się w bloku i co one testują.

W tym przykładzie sprawdzana jest asocjacja między obiektem klasy \emph{User} a wydarzeniem. Chcemy aby jedno wydarzenie należało do jednego użytkownika, dlatego deklarujemy zmienną \emph{user}, która jest instancją\footnote{Obiekt stworzony na podstawie danej klasy.} klasy \emph{User}. Następnie do pola \emph{user\_id} obiektu \emph{event} przypisujemy numer identyfikacyjny(id) obiektu \emph{user}.

\begin{code}
	\lstinputlisting[linerange={43-44}, firstnumber=5]{../meetspace/spec/models/event_spec.rb}
\end{code}\\

Na koniec oczekujemy, że obiekt \emph{event} będzie \emph{,,należał do''} obiektu \emph{user}, oraz, że w polu \emph{event.user} będzie dokładnie ten sam, zadeklarowany kilka linijek wyżej, obiekt.

\begin{code}
	\lstinputlisting[linerange={46-47}, firstnumber=8]{../meetspace/spec/models/event_spec.rb}
\end{code}\\

Całość jest napisana w taki sposób, że nawet osoba nie wdrożona w projekt jest w stanie przeczytać i zrozumieć co dany test robi.
\\

W następnym przykładzie będziemy testować czy przy zadanych przez nas warunkach obiekt zostanie poprawnie zapisany w bazie.

\begin{code}
	\lstinputlisting[linerange={51-63, 100-100}, firstnumber=1]{../meetspace/spec/models/event_spec.rb}
\end{code}\\

W pierwszym teście ustawiamy pole \emph{name} jako nil\footnote{Wartość pusta. Jak \emph{null} w JavaScript.} i oczekujemy, że obiekt nie zostanie poprawnie zwalidowany.

Drugi test sprawdza poprawną długość pola \emph{name}, które powinno wynosić nie więcej niż 50 znaków. Dlatego do zmiennej wstawiamy losowy tekst składający się z większej ilości.\\

Te dwa testy stanowią przykład, że można pisać testy, idąc tzw. ,,czerwoną ścieżką'', czyli oczekiwać, że coś się nie powiedzie.

Ponieżej przedstawiony jest rezultat wszystkich testów jednostkowych.

\begin{figure}[h]
  \centering
  \includegraphics[scale=0.8]{images/rspec_result.png}
  \caption{Wyniki końcowe testów jednostkowych.}
\end{figure}

