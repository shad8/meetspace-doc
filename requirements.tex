\subsection{Wymagania projektowe}
Poniżej przedstawione zostały założenia i~kryteria przyjęte odnośnie aplikacji.
Identyfikują one jej niezbędne cechy, funkcjonalność oraz jakość.
Wymagania zostały podzielone na: funkcjonalne, niefunkcjonalne oraz zgodności.

\subsubsection{Wymagania funkcjonalne}
Wymagania funkcjonalne dotyczą usług, jakie gem ma oferować przyszłym użytkownikom.

\begin{itemize}
  \item Reprezentacja danych w~postaci tabel oraz wykresów słupkowych.
  \item Rozróżnienie użytkowników unikalnych od tych, którzy odwiedzają witrynę po raz kolejny.
  \item Zapisywanie danych dotyczących:
    \begin{itemize}
      \item nazw i~wersji przeglądarek,
      \item informacji o~zainstalowanych pluginach,
      \item nazw i~wersji robotów sieciowych zbierających informacje o~witrynie,
      \item adresów stron internetowych będących refererami,
      \item wyszukiwarek internetowych,
      \item słów kluczowych i~fraz użytych podczas wyszukiwania witryny,
      \item posiadanego przez klienta systemu operacyjnego,
      \item adresów IP oraz Proxy,
      \item rozdzielczości ekranu, okna przeglądarki, głębokości kolorów.
      % \item geolokalizacji użytkownika.
  \end{itemize}
  \item Segregacja danych z~uwzględnieniem:
    \begin{itemize}
      \item godzin,
      \item dni,
      \item dni tygodnia,
      \item miesięcy,
      \item lat.
    \end{itemize}
  \item Sortowanie wyników względem ich wielkości.
  \item Autoryzowany dostęp do raportów.
  \item Analiza danych wywoływana cyklicznie oraz w~czasie przeglądania raportu.
  \item Kompatybilność z~popularnymi przeglądarkami internetowymi.
  % \item Możliwość rozbudowy baz danych

  % \item Działać jako Middleware.
\end{itemize}

\subsubsection{Wymagania niefunkcjonalne}
Wymagania niefunkcjonalne reprezentują zakres działania, wydajność czy też niezawodność gemu.
\begin{itemize}
  \item Działanie przy każdej aplikacji Ruby on Rails od wersji 4.
  % \item Nie obciążanie serwera.
  \item Niezależność od zewnętrznych źródeł zwana \emph{Self Hosted}\index{Self Hosted}\footnote{Typ aplikacji działający na własnej infrastrukturze serwerowej.}.
  \item Instalacja gemu bez potrzeby konfiguracji.
  % \item Większa szybkość działania od rozwiązań opartych na logach
  \item Hermetyzacja kodu i~wyglądu aplikacji.
  \item Uwzględnienie użytkowników nie posiadających JavaScript.
  % \item Aplikacja nie powinna wydłużać czasu ładowania strony. o~ile

\end{itemize}

\subsubsection{Wymagania zgodności}
Wymagania zgodności określają normy i~zasady, w~oparciu o~które ma działać gem.
\begin{itemize}
  \item Opieranie się na licencji X11\footnote{Licencja X11, nazywana także MIT, zezwala na nieograniczone prawo do kopiowania, edycji czy rozpowszechniania. Jednak należy zachować, we wszystkich wersjach, informacje o~autorach i~warunkach licencji. [\ref{license}]}.
  \item Zgodność ze standardami W3C\footnote{World Wide Web Consortium, społeczność zajmująca się tworzeniem standardów internetowych.}.
\end{itemize}
