Meetspace to prosta aplikacja do zarządzania wydarzeniami, której głównym celem jest umożliwienie, jej przyszłym użytkownikom, uzyskania szerszego dostęp do informacji i edukacji w ośrodkach lokalnych.


Jako programiści webowi postawiliśmy przed sobą zadanie, aby stworzyć serwis umożłiwiający wszystkim tym, którzy organizują różnego rodzaju konferencje, spotkania czy warsztaty, na rozreklamowanie się i dotarcie do większego grona odbiorców, całkowicie bezpłatnie. Postanowiliśmy dostarczyć użytkownikom intuicyjny interfejs.  Logowanie i rejestrację do aplikacji poprzez portal Facebook, nie zapominająć o użytkownikach nie posziadających takiego konta. Tworzenie, edytowanie i usuwanie wydarzeń. Możliwość spersonalizowania ich strony do własnych wymagań.


Zdecydowaliśmy się rówież na zbudowanie systemu pozwalającego na rozsyłanie najświeższych informacji na temat publikacji wszystkim osobą subskrybującym naszą aplikację. Przez co, w każdą niedzielę na ich skrzynkach pocztowych pojawi się lista z ostatnio dodanymi wydarzeniami.

Podczas tworzenia projektu postanowiliśmy rówież wykorzystać najnowsze standardy tworzenia oprogramowania. Zastosować elementy metodyki Scram.

Aby spełnić wyżej wymienione założenia wykorzystaliśmy język Ruby i framework Ruby on Rails. Wybraliśmy te technologie ze względu na szybkość i łatwość tworzenia oprogramowania, oraz dostępność wielu pomocnych narzędzi i gotowych rozwiązań. Wszystko po to, aby dostarczyć użytkownikowi funkcjonalny i łatwy w obsłudze produkt.
