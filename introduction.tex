\textbf{Meetspace} to prosta aplikacja do zarządzania wydarzeniami, której głównym celem jest umożliwienie, jej przyszłym użytkownikom uzyskania szerszego dostępu do informacji i edukacji w ośrodkach lokalnych.


Jako programiści webowi postawiliśmy przed sobą zadanie, aby stworzyć serwis umożliwiający wszystkim tym, którzy organizują różnego rodzaju konferencje, spotkania czy warsztaty, na rozreklamowanie się i dotarcie do większego grona odbiorców, całkowicie bezpłatnie.
Postanowiliśmy dostarczyć użytkownikom intuicyjny interfejs, umożliwiający logowanie i rejestrację do aplikacji poprzez portal Facebook, nie zapominając także o użytkownikach nie posiadających takiego konta. Zapewniliśmy tworzenie, edytowanie i usuwanie wydarzeń, a także
możliwość spersonalizowania ich strony do własnych wymagań.


Zbudowaliśmy system mailingowy pozwalający na rozsyłanie najświeższych informacji na temat publikacji wszystkim osobom korzystających z newslettera aplikacji. W każdą niedzielę na skrzynkach pocztowych subskrybentów pojawi się lista z ostatnio dodanymi wydarzeniami przez organizatorów. Ułątwi to przepływ inforamcji i pozwoli na zwiększenie ilości słuchaczy czy osób zainteresowanych daną inicjatywą.

Podczas tworzenia projektu postanowiliśmy kierować się najnowszymi standardami tworzenia oprogramowania, wykorzystując wzorce projektowe czy zasadę \index{SOLID}SOLID\footnote{SOLID - pięć podstawowych założeń programowania obiektowego: Single responsibility, Open-closed, Liskov substitution, Interface segregation and Dependency inversion,  stworzonych przez Roberta C. Martina}. Aby spełnić wyżej wymienione założenia urzyliśmy języka \index{Ruby} Ruby\footnote{Więcej o języku Ruby w rozdziale \ref{sec:Ruby}} i framework \index{Ruby on Rails} Ruby on Rails\footnote{Więcej o Ruby on Rails w rozdziale \ref{sec:Rails}}. Wybraliśmy te technologie ze względu na szybkość i łatwość z jaką powstaje dzięki nim oprogramowanie, a także zw względu na dostępność wielu pomocnych narzędzi i gotowych rozwiązań. Technologie te mają bardzo dobrze rozwiniętą dokumentację, co pozwala w krótkim czasie na znajdowanie odpowiedzi na nurtujące pytania z danego zagadnienia. Ponadto tworząc aplikację zastosowaliśmy elementy metodyki zwinnego wytwarzania oprogramowania, takie jak testy czy metryki świadczące o jakości kodu. Wybrane technologie w dużej mierze ułątwiły wprowadzanie narzuconych sobie założeń.
Wszystko po to, aby dostarczyć użytkownikowi funkcjonalny i łatwy w obsłudze produkt jak najszybciej.
