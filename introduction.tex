\textbf{Meetspace} to prosta aplikacja do zarządzania wydarzeniami, której głównym celem jest umożliwienie, jej przyszłym użytkownikom, uzyskania szerszego dostęp do informacji i edukacji w ośrodkach lokalnych.


Jako programiści webowi postawiliśmy przed sobą zadanie, aby stworzyć serwis umożliwiający wszystkim tym, którzy organizują różnego rodzaju konferencje, spotkania czy warsztaty, na rozreklamowanie się i dotarcie do większego grona odbiorców, całkowicie bezpłatnie.
Postanowiliśmy dostarczyć użytkownikom intuicyjny interfejs.
Logowanie i rejestrację do aplikacji poprzez portal Facebook, nie zapominając o użytkownikach nie posiadających takiego konta. Tworzenie, edytowanie i usuwanie wydarzeń.
Możliwość spersonalizowania ich strony do własnych wymagań.


Zbudowaliśmy system mailingowy pozwalającego na rozsyłanie najświeższych informacji na temat publikacji wszystkim osobom zapisanych na newsletter aplikacji.
W każdą niedzielę na skrzynkach pocztowych subskrybentów pojawi się lista z ostatnio dodanymi wydarzeniami.

Podczas tworzenia projektu postanowiliśmy również wykorzystać najnowsze standardy tworzenia oprogramowania.
Zastosowaliśmy elementy metodyki Scrum.

Aby spełnić wyżej wymienione założenia wykorzystaliśmy język Ruby i framework Ruby on Rails.
Wybraliśmy te technologie ze względu na szybkość i łatwość tworzenia oprogramowania, oraz dostępność wielu pomocnych narzędzi i gotowych rozwiązań.
Wszystko po to, aby dostarczyć użytkownikowi funkcjonalny i łatwy w obsłudze produkt jak najszybciej.
