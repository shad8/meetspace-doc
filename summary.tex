\dodatkowo{Podsumowanie}
Praca nad projektem sporo nas nauczyła, poszerzając wiedzę nie tylko z programowania, ale również bezpieczeństwa i~komunikacji.
Poznaliśmy sporo nowych technologii oraz rozwiązań, dzięki którym udało nam się stworzyć aplikację będącą osiągnięciem wszystkich celów, które przed sobą postawiliśmy.
Wierzymy, że \textbf{Meetspace} pomoże wielu osobom w organizowaniu i wyszukiwaniu nowych wydarzeń w okolicy.

Podczas dwumiesięcznej pracy udało nam się zrealizować zaplanowane funkcjonalności.
Aplikacja umożliwia organizatorom wydarzeń ich tworzenie i modyfikowanie.
Użytkownicy mogą w prosty sposób dołączać do społeczności meetspace poprzez logowanie za pomocą portalu Facebook. Korzystać z prostego i intuicyjnego mechanizmu wyszukiwania wydarzeń.
Zastosowany responsywny layout pozwala na korzystanie z aplikacji zarówno na komputerach stacjonarnych jak i urządzeniach mobilnych takich jak tablety czy smartfony.
Ponadto udało nam się również zaprojektować i zaimplementować prosty interfejs komunikacyjny między aplikacjami pozwalający na pobieranie informacji o wydarzeniach z naszej aplikacji.
Dzięki zastosowaniu takiego rozwiązania umieszczane informacje mogą być udostępniane na innych portalach internetowych czy urządzeniach mobilnych.

Wykorzystanie testów oraz stosowanie się do odpowiednich metryk dla języka Ruby, przyczyniło się do powstania dobrej jakości kodu.
Jest on prosty, czytelny i łatwy w interpretacji.
Dzięki temu, kolejne osoby, wdrażające się w projekt, nie będą miały trudności z zapoznaniem się z działaniem aplikacji.

Meetspace jest otwarta na rozwój.
Został zaprojektowany z myślą o dodawaniu do niego nowych funkcjonalności.
Pragniemy wyjść na przeciw potrzebom przyszłych użytkowników.
W przyszłości planujemy z wykorzystaniem zaimplementowanego API stworzyć proste widget na smartfony wyświetlające najnowsze wydarzenia.
Dodać możliwość publikacji wydarzeń na profilu Facebook i zaprojektować system przypominania o wydarzeniach poprzez wiadomości SMS.
