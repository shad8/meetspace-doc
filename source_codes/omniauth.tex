\begin{code}
  \lstinputlisting[language = Ruby, linerange={234-234}, firstnumber = 234]{../meetspace/config/initializers/devise.rb}
\end{code} \\

API ID I API SECRET zostały zapisane, ze względów bezpieczeństwa, w specjalnym pliku \emph{.env}\footnote{Plik przechowywujący zmienne środowiskowe, więcej informacji w rozdziale o bezpieczeństwie aplikacji.} pod zmiennymi \texttt{FACEBOOK\_KEY} i \texttt{FACEBOO\_KSECRET}. \\

Poniżej został przedstawiony kontroler obsługujący logowanie, który dziedziczy wymagane metody zaimplementowane w module gemu \emph{Devise}. \\

\begin{code}
  \lstinputlisting[language = Ruby]{../meetspace/app/controllers/authenitications_controller.rb}
\end{code} \\

\clearpage
Dwoma najważniejszymi parametrami, które identyfikują użytkownika logującego się za pomocą portalu jest:
\begin{itemize}
   \item \texttt{uid} - unikalny identyfikator użytkownika w danym portalu społecznościowym
   \item \texttt{provider} - przechowuje nazwę portalu społecznościowego
 \end{itemize}
zawarte w \texttt{request.env['omniauth.auth']}.
Z wykorzystaniem tych parametrów jest tworzone lub odnajdywane w bazie konto użytkownika. \\

\begin{code}
  \lstinputlisting[language = Ruby]{../meetspace/app/models/authentication.rb}
\end{code}
