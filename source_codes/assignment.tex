Poniższy przykładowy atak przypisuje stworzone wydarzenie przypadkowemu użytkownikowi.

\texttt{ \footnotesize curl -d "event[name]=Test\&event[user\_id]=3" http://meetspace.it/event/ }

Rozwiązaniem tego problemu jest zastosowanie techniki tzw. \emph{Strong Parameters}. Uniemożliwia ona przesłanie zmodyfikowanych parametrów, ponieważ kontroler wybiera jedynie dane dozwolone. Niedozwolone parametry zostają zignorowane. Poniżej przedstawiono przykładowe użycie tej metody dla kontrolera zarządzającego wydarzeniami.

\begin{code}
  \lstinputlisting[language=Ruby, linerange={38-42}]{../meetspace/app/controllers/events_controller.rb}
\end{code}\\
\index{Interaktor}
Zastosowanie wzorca projektowego interaktora również uniemożliwia tego rodzaju atak. Podobnie jak w powyższym przykładzie, interaktor przyjmuje jedynie zdefiniowane parametry.

\begin{code}
  \lstinputlisting[language=Ruby, linerange={1-6}]{../meetspace/app/interactions/subscriber_registration_update.rb}
\end{code}\\
