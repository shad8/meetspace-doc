Aby móc z niego korzystać, należy do kodu strony dołączyć odpowiedni skrypt\footnote{Program wykonywalny wewnątrz aplikacji} JavaScript.
\begin{code}
  \begin{lstlisting}[language=Ruby, basicstyle=\ttfamily\scriptsize, showstringspaces=false]
      <script src='https://maps.googleapis.com/maps/api/js?v=3.exp&libraries=places'>
      </script>
  \end{lstlisting}
\end{code}\\

a następnie w kodzie HTML utworzyć kontener przechowywujący mapę.
\begin{code}
  \lstinputlisting[language=HTML, showstringspaces=false, linerange={46-48}, firstnumber=42]{../meetspace/app/views/events/show.html.erb}
\end{code}\\

Aby poprawnie wyświetlić mapę potrzebujemy współrzędnych adresu miejsca, w którym będzie się odbywać spotkanie. Gem \emph{Geolocalizer} na podstawie podanego przez użytkownika adresu, zwraca nam współrzędne, które musimy przekazać do skryptu.

\begin{code}
  \lstinputlisting[showstringspaces=false, basicstyle=\ttfamily\scriptsize]{../meetspace/app/assets/javascripts/map.js.coffee}
\end{code}\\

Następnie ustawiamy opcje mapy, środek, wielkość powiększenia oraz typ. Mamy do wyboru mapę drogową, satelitarną lub połączenie tych dwóch. Powiększenie można wybrać z zakresu od 0 do 21. Im większa wartość, tym więcej szczegółów jest widocznych. 

Na koniec tworzymy dwa obiekty na podstawie udostępnionych przez API konstruktorów: \emph{map} oraz \emph{marker}. \emph{Map} przyjmuje dwa parametry: miejsce na stronie, w którym będzie wyświetlana oraz opcje z jakimi zostanie zainicjalizowana.
Natomiast \emph{marker} to znacznik wskazujący na dokładny adres. Jako parametr przyjmuje współrzędne wskazywanego miejsca. 
