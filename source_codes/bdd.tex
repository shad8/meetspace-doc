Pierwszym krokiem jest opisanie w kilku słowach funkcjonalności.

\begin{code}
	\lstinputlisting[linerange={0-4}]{../meetspace/features/events.feature}
\end{code}\\

Następnie tworzymy scenariusz, w którym krok po kroku będziemy wykonywać czynności aż otrzymamy efekt końcowy. W tym przypadku będzie to wyświetlenie nowo dodanego wydarzenia.

Słowem kluczowym ,,Scenario'' nadajemy tytuł nowego scenariusza. Jest to o tyle pomocne, że w momencie uruchomienia testów, widać który scenariusz jest w tej chwili testowany.

\begin{code}
	\lstinputlisting[linerange={10-10}, firstnumber=5]{../meetspace/features/events.feature}
\end{code}\\

Słowem ,,Given'' definiujemy stan początkowy aplikacji, czyli miejsce w aplikacji, w którym będziemy zaczynać wykonywanie konkretnych czynności, w tym przypadku jest to strona główna oraz zalogowanie się. Zalogowanie się jest istotnym elementem, ponieważ, bez tego nie będziemy w stanie dodać żadnego nowego wydarzenia.

\begin{code}
	\lstinputlisting[linerange={7-8}, firstnumber=6]{../meetspace/features/events.feature}
\end{code}\\

Następnie trzeba kliknąć przycisk ,,Utwórz wydarzenie'', wypełnić jego informacje: nazwę, datę, datę zakończenia, czas rozpoczęcia, plan i logo.

\begin{code}
	\lstinputlisting[linerange={11-20}, firstnumber=8]{../meetspace/features/events.feature}
\end{code}\\

Po słowie ,,Then'' definiujemy nasze oczekiwania. W tym przypadku chcemy zobaczyć nowo dodane wydarzenie wraz z jego wszystkimi właściwościami. Słowa ,,And'' używamy jeśli chcemy rozwinąć listę wykonywanych kroków lub listę oczekiwań.

\begin{code}
	\lstinputlisting[linerange={21-26}, firstnumber=18]{../meetspace/features/events.feature}
\end{code}\\

W ten sposób mamy opisaną, za pomocą testu integracyjnego, całą ścieżkę, którą musi przejść użytkownik aby stworzyć nowe wydarzenie.

W przypadku gdy chcemy przetestować edytowanie istniejącego już wpisu, nie musimy przechodzić procesu tworzenia od początku. Wtedy wystarczy, że na wstępie stworzymy już gotowe wydarzenie, które w następnych krokach będzie modyfikowane.

\begin{code}
	\lstinputlisting[linerange={28-31}, firstnumber=1]{../meetspace/features/events.feature}
\end{code}\\

W tej chwili w bazie testowej mamy utworzony jeden rekord. Dane z jakimi tworzone jest wydarzenie nie mają w tym momencie najmniejszego znaczenia, nie to jest tutaj testowane. Równie dobrze można by wpisać dowolny ciąg znaków, ale kierujemy się dobrą praktyką programistyczną i staramy się pisać zrozumiały kod.

Aby móc edytować jakikolwiek wpis, musimy wejść na jego stronę i kliknąć odpowiedni przycisk, bądź link, żeby przejść do strony z formularzem.

\begin{code}
	\lstinputlisting[linerange={32-33}, firstnumber=3]{../meetspace/features/events.feature}
\end{code}\\

Zmieniamy wpisane wartości na ,,Party'' oraz ,,15:00 Start'' oraz ustawiamy nowe logo.

\begin{code}
	\lstinputlisting[linerange={34-37}, firstnumber=5]{../meetspace/features/events.feature}
\end{code}\\

I na koniec oczekujemy, że wprowadzone przed chwilą zmiany zobaczymy na stronie wydarzenia.

\begin{code}
	\lstinputlisting[linerange={38-39}, firstnumber=9]{../meetspace/features/events.feature}
\end{code}\\

Można łatwo zauważyć, że część kroków powtarza się w pierwszym jak i w drugim scenariuszu. Jest to nic innego jak zastosowanie metody DRY\footnote{Don't repeat yourself - Nie powtarzaj się}. Dzięki temu nowe testy powstają coraz szybciej i sprawniej.
