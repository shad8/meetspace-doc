Przykładowy atak SQL Injection przy poniższym zapisie wyszukiwania wydarzeń, daje atakującemu możliwość wpisania dowolnego parametru do zapytania.

\begin{center}
  \texttt{ Event.where("name LIKE \%\#\{params[:name]\}") }
\end{center}

Dla wprowadzenia danych \emph{OR 1=1} otrzymamy następujace zapytanie do bazy,

\begin{center}
  \texttt{ SELECT * FROM events WHERE name = '' OR 1 --' }
\end{center}

które spowoduje wyświetlenie wszystkich wydarzeń, również tych niedostępnych dla użytkowników. Przyczyną takiego wyniku jest znak komentarza \emph{- -} na końcu zapytania uniemożliwiający wykonanie dalszej część zapytania.
Sposobem na zapewnienie bezpieczeństwa jest w tym przypadku stosowanie odpowiedniego zapisu zapytań:

\begin{code}
  \lstinputlisting[language=Ruby, linerange={7-10}]{../meetspace/app/interactions/event_search.rb}
\end{code}\\

W miejsca, gdzie znajdują się znaki zapytania, zostają wstawione wartości ze zmiennej \emph{search}. Zawartość zmiennej zostaje sprawdzona przez wbudowane filtry wykrywające znaki specjalne dla zapytań SQL.
