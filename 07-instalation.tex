\section{Instalacja}
Poniżej zostały przedstawione najważniejsze elementy dotyczące instalacji i konfiguracji aplikacji dla środowiska produkcyjnego. Aplikacja Meetspace jest umieszczona na serwerze firmy Fractal Soft\footnote{\url{http://fractalsoft.org/}}, z którą mieliśmy okazję współpracować.

Przedstawiony przykład został wykonany na systemie operacyjnym Linux, dystrybucja: Ubuntu Serwer 12.04, z zainstalowanym system kontroli wersji GIT\footnote{Więcej o systemie GIT w rozdziale \ref{sec:GIT}}, narzędziem RVM\footnote{Więcej o narzędziu RVM w rozdziale: \ref{sec:RVM}} i bazą danych MySQl.

Instalacja RVM:
\begin{center}
  \texttt{ curl -L get.rvm.io | bash -s stable }
  \texttt{ rvm requirements }
\end{center}

Instalacja GIT:
\begin{center}
  \texttt{ sudo apt-get install git-core }
\end{center}

Instalacja bazy danych MySQL:
\begin{center}
  \texttt{ sudo apt-get install mysql-server mysql-client libmysqlclient-dev }
\end{center}

Aby przygotować wersję użytkową aplikacji, należy zainstalować środowisko Ruby wykorzystując do tego celu RVM:
\begin{center}
  \texttt{ rvm install 2.1.3 }
\end{center}
Do środowiska należy również dołączyć interpreter JavaScript:
\begin{center}
  \texttt{ sudo apt-get install nodejs }
\end{center}

\clearpage
Jednym z ważniejszych kroków jest instalacja serwera www \emph{Nginx}, wsperającego język Ruby.
\begin{center}
  \texttt{ sudo apt-get install nginx }
\end{center}

Do sprawnego zarządzania serwerem warto doinstalować \emph{Unicorn}, który w łatwy sposób umożliwia zarządzanie procesami dotyczącymi serwera www, np. start lub restart.
\begin{center}
  \texttt{ gem install unicorn }
\end{center}

Po odpowiednim skonfigurowaniu Nginx, Unicorna i aplikacji, poprzez polecenie
\begin{center}
  \texttt{ ./bin/unicorn\_init start }
\end{center}

uruchamiamy aplikację na serwerze.
