\section{Opis projektu}
W tym rozdziale zamieściliśmy informacje dotyczące genezy aplikacji Meetspace. Opisaliśmy jej przeznaczenie, problemy użytkowników do których zostanie skierowana, przegląd konkurencyjnych aplikacji i wymagania jakie przed sobą postawiliśmy.
\subsection{Problem}
W wielu miastach odbywa się bardzo dużo różnego rodzaju lokalnych konferencji, szkoleń i warsztatów. Wiele z nich jest całkowicie bezpłatnych, przez co organizatorzy mają ograniczoną pulę środków przeznaczonych na ich reklamę. Z tego powodu informacje na ich temat są rozproszone po portalach społecznościowych czy aplikacjach internetowych. Lokalnym społecznościom brakuje scentralizowanego punktu dostępu do informacji, a organizatorom wydarzeń miejsca na ich zamieszczanie. Problem ten jest bardzo często widoczny wśród środowiska studenckiego, gdzie wydawałoby się, że dostęp do informacji nie stanowi problemu.  
\subsection{Meetspace}
Aplikacja Meetspace umożliwia w łatwy sposób tworzenie i zarządzanie wydarzeniami przy pomocy narzędzi online. Stanowi doskonałe narzędzie do budowania lokalnych przedsięwzięć. Ponadto, użytkownicy mogą wyszukiwać wydarzenia za pomocą takich filtrów jak: lokalizacja, termin czy tematyka, a także dzielić się nimi w~ sieci poprzez portale społecznościowe. Aplikacja posiada bardzo prosty i przyjazny dla użytkownika interfejs. Meetspace został wyposażony również we własny system API, który umożliwia pobieranie i umieszczanie informacji o wydarzeniach na innych stronach internetowych i aplikacjach mobilnych. W przyszłości może on posłużyć do stworzenia dedykowanej aplikacji na telefony komórkowe. Dzięki serwisowi uzyskuje się zatem szerszy dostęp do informacji i edukacji w ośrodkach lokalnych.
\subsection{Zestawienie konkurencyjnych aplikacji}
Na rynku istnieje wiele aplikacji wykorzystywanych do publikowania wydarzeń. Postanowiliśmy przeanalizować działanie tych najbardziej popularnych, sprawdzić co dane aplikacje oferują użytkownikom, czego im brakuje, w jaki sposób został u~ nich zbudowany interfejs graficzny użytkownika i stworzyć aplikację, która wypełni lukę i rozwiąże wyżej przedstawiony problem.

Analiza została przeprowadzona na pięciu wybranych aplikacjach internetowych:
\begin{itemize}
  \item \href{http://www.meetup.com/}{Meetup} \\
  Jedna z najbardziej popularnych aplikacji, z której mieliśmy przyjemność wielokrotnie korzystać. Skupia ona ludzi w grupy o podobnych zainteresowaniach, gdzie można wyrażać opinie i umieszczać informacje. Dostarcza ona informacji o wydarzeniach z różnych lokalizacji poprzez uczestnictwo w danej grupie. Taka opcja nie jest jednak darmowa. Meetup umożliwia tworzenie informacji o wydarzeniu jedynie poprzez grupę. Aplikacja posiada przyjazny interfejs, który nie jest przystosowany do urządzeń mobilnych.
  \item \href{https://www.eventbrite.com/}{Eventbrite} \\
  Eventbrite jest bardzo rozbudowaną aplikacją, przez co proces tworzenia wydarzenia jest czasochłonny. Wydarzenia umieszczane na Eventbrite mają zasięg globalny i wyrobioną markę. Aplikacja nakierowana jest na dystrybucję biletów na opublikowane wydarzenie.
  \item \href{http://busyconf.com/}{BusyConf} \\
  Zaawansowana aplikacja przeznaczona dla organizatorów wydarzeń. Obsługuje jedynie język angielski. Posiada bardzo wiele narzędzi wspomagających organizację wydarzenia. Nie jest jednak prosta w obsłudze a interfejs nie jest przyjazny dla początkujących użytkowników.
  \item \href{http://www.uczestnicy.pl/}{uczestnicy.pl} \\
  Jedna z lepszych polskich aplikacji jakie udało nam się znaleźć lecz o słabej popularności. Przeznaczona głównie do tworzenia i publikacji informacji o wydarzeniach. Posiada niezbyt dobrze funkcjonujący system filtrowania wydarzeń. Umieszczanie wydarzeń wiąże się z poniesieniem dodatkowych opłat.
  \item \href{http://sk.polsl.pl/}{sk.polsl.pl} \\
  Strona Politechniki Śląskiej, której celem jest informowanie o konferencjach. Zamieszczane wiadomości nie przekazują istotnych informacji. Jest ona mało czytelna i nieintuicyjna dla użytkownika. Jedynym dostępnym językiem jest język angielski.
\end{itemize}


Wyniki analizy zostały zebrane i przedstawione w postaci tabeli. Pokazuje ona, że żadna z aplikacji nie rozwiązuje w pełni postawionego wcześniej problemu. Część z nich służy jedynie jako narzędzie wspomagające organizację wydarzenia bez jego późniejszej reklamy. Inne aplikacje natomiast umożliwiają informowanie uczestników po dokonaniu odpowiedniej opłaty. Jedynie aplikacja Meetup udostępnia możliwość zapisania się do newslettera i otrzymywania powiadomień o nadchodzących wydarzeniach na pocztę e-mail.

\newcommand{\cmark}{\ding{51}}
\newcommand{\xmark}{\ding{55}}

\begin{landscape}
  \begin{table}[h]
    \begin{tabular}{|l|c|c|c|c|c|c|}
      \hline
      Funkcjonalność                          & Meetup & Eventbrite & BusyConf & uczestnicy.pl & sk.polsl    & \textbf{Meetspace} \\ \hline
      wielojęzyczność                         & \cmark & \cmark     & \xmark   & \xmark        & \xmark      & \cmark             \\ \hline
      responsuwność                           & \xmark & \cmark     & \cmark   & \cmark        & \xmark      & \cmark             \\ \hline
      przyjazny interfejs                     & \cmark & \cmark     & \cmark   & \xmark        & \xmark      & \cmark             \\ \hline
      logowanie przez portale społecznościowe & \cmark & \xmark     & \xmark   & \cmark        & \xmark      & \cmark             \\ \hline
      lokalizacja wydarzeń na mapie           & \xmark & \cmark     & \cmark   & \cmark        & \xmark      & \cmark             \\ \hline
      newsletter                              & \cmark & \xmark     & \xmark   & \xmark        & \xmark      & \cmark             \\ \hline
      brak opłat                              & \xmark & \cmark     & \xmark   & \xmark        & brak danych & \cmark             \\ \hline
      własne API                              & \cmark & \xmark     & \xmark   & \xmark        & \xmark      & \cmark             \\ \hline
    \end{tabular}
    \caption{Konkurecyjne aplika - zestawienie wybranych funkcjonalności, dane z dnia 20-10-2014}
  \end{table}
\end{landscape}


\subsection{Wymagania projektowe}
W tym rozdziale umieściliśmy zbiór wymagań i funkcjonalności dotyczących naszej aplikacji.
  \subsubsection{Wymagania funkcjonalne}
    Wymagania funkcjonalne są to wszystkie funkcjonalności jakie będzie oferować aplikacja Meetspace swoim przyszłym użytkownikom.
    \begin{itemize}
      \item Aplikacja umożliwia organizatorom tworzenie nowych wydarzeń oraz zarządzanie nimi.
      \item Użytkownicy mają możliwość zapisania się do newslettera, dzięki któremu raz w tygodniu otrzymają informacje o nadchodzących wydarzeniach.
      \item Możliwość zalogowania się w aplikacji poprzez konto na portalu społecznościowym Facebook.
      \item Dodanie mapy z adresem miejsca, w którym dane wydarzenie będzie się odbywać.
    \end{itemize}
  \subsubsection{Wymagania niefunkcjonalne}
  Wymagania niefunkcjonalne są to założenia reprezentujące niezawodność i zakres działania aplikacji.
    \begin{itemize}
      \item Wykorzystanie technologii Ruby i Ruby on Rails.
      \item Obsługa dwóch języków, polskiego oraz angielskiego.
      \item Obsługa najpopularniejszych przeglądarek internetowych.
      \item Uwzględnienie użytkowników nie korzystających z JavaScript.
    \end{itemize}

  \subsubsection{Wymagania zgodności}
    Podczas projektowania aplikacji stosowaliśmy przyjęte normy i zasady. Kierując się nimi uzyskaliśmy czytelny kod o wysokiej jakości, nie zawierający duplikacji i~ zbędnych funkcjonalności.
    \begin{itemize}
      \item Zgodność ze standardami W3C\footnote{World Wide Web Consortium, społeczność zajmująca się tworzeniem standardów internetowych.}.
      \item Definition of Done\footnote{Publicznie dostępny standard firmy Fractal Soft\cite{dod}}.
      \item Simple Coding Standard\footnote{Publicznie dostępny standard firmy Fractal Soft\cite{scs}}.
      \item The Ruby Style Guide\footnote{Ogólnie przyjęte zasady programowania w języku Ruby \cite{ruby_style_guide}}.
      \item The Rails Style Guide\footnote{Ogólnie przyjęte zasady programowania w Ruby on Rails \cite{rails_style_guide}}.
    \end{itemize}
