\section{Opis projektu}

\subsection{Problem}
\subsection{Meetspace}
\subsection{Zestawienie podobnych aplikacji}
\subsection{Wymagania projektowe}
W tym rozdziale umieściliśmy zbiór wymagań i funkcjonalności dotyczących naszej aplikacji.
  \subsubsection{Wymagania funkcjonalne}
    \begin{itemize}
      \item Aplikacja umożliwia tworzenie organizatorom nowych wydarzeń oraz zarządzanie nimi.
      \item Użytkownicy mają możliwość zapisania się do newslettera, dzięki któremu raz w tygodniu otrzymają informacje o nadchodzących wydarzeniach.
      \item Możliwość zalogowania się w aplikacji poprzez konta portali społecznościowych (Facebook).
      \item Dodanie mapy z adresem miejsca, w którym dane wydarzenie będzie się odbywać.
    \end{itemize}
  \subsubsection{Wymagania niefunkcjonalne}
    \begin{itemize}
      \item Obsługa dwóch języków, polski oraz angielski
      \item Intuicyjny interfejs
      \item Obsługa najpopularniejszych przeglądarek internetowych.
      \item Uwzględnienie użtkowników nie korzystających z JavaScript.
    \end{itemize}

  \subsubsection{Wymagania zgodności}
    Podczas projektowania aplikaji stosowaliśmy przyjęte normy i zasady.
    \begin{itemize}
      \item Zgodność ze standardami W3C\footnote{World Wide Web Consortium, społeczność zajmująca się tworzeniem standardów internetowych.}.
      \item Definition of done\footnote{Dostępne pod adresem github.com}.
      \item Simple coding standard\footnote{Fractalsoft.org}
      \item The Ruby Style Guide\footnote{Ogólno przyjęte zasady programowania w języku Ruby \cite{ruby_style_guide}}.
      \item The Rails Style Guide\footnote{Ogólno przyjęte zasady programowania w Ruby on Rails \cite{rails_style_guide}}.
    \end{itemize}
